\documentclass[a4paper, 10pt]{article}

%% Language and font encodings
\usepackage[english]{babel}
\usepackage[utf8]{inputenc}
\usepackage[T1]{fontenc}

%% Sets page size and margins
\usepackage[a4paper,top=2.5cm,bottom=2cm,left=2.5cm,right=1.9cm]{geometry}


%% Useful packages
\usepackage{amsmath}
\usepackage{siunitx}
\usepackage{booktabs}
\usepackage{graphicx}
\usepackage[colorinlistoftodos]{todonotes}
\usepackage[colorlinks=true, allcolors=blue]{hyperref}
\usepackage{hyperref}
\usepackage{subcaption}
\usepackage{multirow}
\usepackage{arydshln}  % dashed lines

\renewcommand{\arraystretch}{1.25} % for tables

\title{Your Paper}
\author{You}

\begin{document}
\maketitle
\newpage
\begin{table}[h!]
    \begin{tabular}{@{}lrrrr@{}}
      \toprule
      \multicolumn{5}{c}{\textbf{gd}} \\
      N  &   R_{t}  &  MSE_{t} &  Epch  & T(s)\\
      \midrule
      10       &   0.06     &  0.49       &  771     & 2.64   \\
      20       &   0.31     &  0.37       &  849     & 2.82   \\
      40       &   0.32     &  0.35       &  915     & 3.58   \\
      80       &   0.49     &  0.27       &  958     & 4.12   \\
      \hdashline
      100      &   0.48     &  0.53       &  976     & 4.90   \\
      120      &   0.47     &  0.62       &  957     & 7.15   \\
      \bottomrule
    \end{tabular} 
    \hfill
    \begin{tabular}{@{}lrrrr@{}}
      \toprule
      \multicolumn{5}{c}{\textbf{gda}} \\
      N  &   R_{t}  &  MSE_{t} &  Epch  & T(s) \\
      \midrule
      10  & 0.17    & 0.51    & 79      &  0.28   \\
      20  & 0.25    & 0.48    & 79      &  0.53   \\
      40  & 0.29    & 0.54    & 67      &  0.58   \\
      80  & 0.22    & 0.88    & 68      &  0.59   \\                 
      \hdashline
      100 & 0.20    & 0.90    & 112     &  0.51   \\
      120 & 0.21    & 1.15    & 97      &  0.47   \\
      \bottomrule
    \end{tabular} 
    \hfill
    \begin{tabular}{@{}lrrrr@{}}
      \toprule
      \multicolumn{5}{c}{\textbf{cgf}} \\
      N  &   R_{t}  &  MSE_{t} &  Epch  & T(s) \\
      \midrule
      10  & 0.15    & 0.50    & 11      &  0.10  \\
      20  & 0.42    & 0.38    & 17      &  0.31  \\
      40  & 0.65    & 0.28    & 30      &  0.54  \\
      80  & 0.69    & 0.29    & 44      &  0.53  \\
      \hdashline
     100  & 0.68    & 0.35    & 37      &  0.59  \\
     120  & 0.60    & 0.45    & 38      &  0.76  \\
      \bottomrule
    \end{tabular} 
    \mbox{}
    \begin{tabular}{@{}lrrrr@{}}
      \toprule
      \multicolumn{5}{c}{\textbf{cgp}} \\
      N  &   R_{t}  &  MSE_{t} &  Epch  & T(s) \\
      \midrule
      10  & 0.17    & 0.48    & 12      &  0.10  \\
      20  & 0.33    & 0.48    & 15      &  0.40  \\
      40  & 0.60    & 0.32    & 25      &  0.48  \\
      80  & 0.67    & 0.31    & 35      &  1.59  \\
      \hdashline
     100  & 0.62    & 0.38    & 40      &  0.66  \\
     120  & 0.62    & 0.44    & 32      &  0.63  \\
      \bottomrule
    \end{tabular} 
    \hfill
    \begin{tabular}{@{}lrrrr@{}}
      \toprule
      \multicolumn{5}{c}{\textbf{bfg}} \\
      N  &   R_{t}  &  MSE_t &  Epch  & T(s) \\
      \midrule
      10  & 0.18    & 0.48   & 11       & 0.15  \\
      20  & 0.43    & 0.40   & 17       & 0.30  \\
      40  & 0.77    & 0.20   & 35       & 0.86  \\
      80  & 0.91    & 0.9    & 39       & 2.34  \\
      \hdashline
     100  & 0.88    & 0.12   & 47       & 2.55  \\   
     120  & 0.75    & 0.28   & 47       & 3.59  \\
      \bottomrule
    \end{tabular} 
    \hfill
    \begin{tabular}{@{}lrrrr@{}}
      \toprule
      \multicolumn{5}{c}{\textbf{lm}} \\
      N  &   R_{t}  &  MSE_t &  Epch  & T(s) \\
      \midrule
      10  & 0.18    & 0.44   & 11       & 0.07  \\ 
      20  & 0.50    & 0.38   &  6       & 0.10  \\ 
      40  & 0.85    & 0.13   & 11       & 0.18  \\ 
      \textbf{80} & \textbf{.93} & \textbf{.07}  & \textbf{6} & 
      \textbf{.31} \\
      \hdashline
     100  & 0.90    & 0.10   & 4        & 0.26  \\
     120  & 0.81    & 0.22   & 5        & 0.27  \\  
      \bottomrule
    \end{tabular} \mbox{}
    \caption{Performance of different training algorithms. Results are the
      average ones after twenty runs on the test set. \emph{N} is the
      neurons on the hidden layer. $R_t$ and $MSE_t$ 
    stand for regression R-value and for MSE on the test set, respectively. 
    Test size is 15\% of the total data. \emph{Epch} stands for epoch of 
    convergence. \emph{T(s)} is time (seconds). Best result is 
    highlighted in bold. The dotted line separates overfitted results  }
    \label{tab:train_algs}

    \section{Algorithm comparison}
    The goal here is comparing the six algorithms for training Neural Nets on
    the Matlab toolbox. For this purpose, I have trained a net using each 
    algorithm with different number of hidden units. Then, I have evaluated
    is performance on the test set, and repeated the process twenty times. 
    \autoref{tab:train_algs} shows the results of these experiments, averaged for
    each algorithm and neuron setting. Overall, all algorithms perform best
    with 80 hidden units: nets with neurons over that threshold 
    start to overfit, since
    their MSE on the test set starts to decrease. The best algorithm in terms 
    of speed and performance is the \emph{Levenberg-Marquardt} (lm) algorithm:
    it trains the fastest, as well as reaching better performance than the other
    ones under the same configurations. \emph{Gradient descent} is by far the
    worst, both in speed and accuracy. The reason is that [maybe that looks for
    the best option/not optimized -check]

  \end{table}




\end{document}

